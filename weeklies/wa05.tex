
\documentclass[12pt]{amsart}
\usepackage{geometry} % see geometry.pdf on how to lay out the page. There's lots.
\geometry{a4paper} % or letter or a5paper or ... etc
\usepackage[T1]{fontenc}
\usepackage[latin9]{inputenc}
\usepackage{amsmath}
\usepackage{amsaddr}
\usepackage{dirtytalk}
\usepackage{float}
\usepackage{listings}
\usepackage{hyperref}
\usepackage{enumerate}

\usepackage{color}
 
\definecolor{codegreen}{rgb}{0,0.6,0}
\definecolor{codegray}{rgb}{0.5,0.5,0.5}
\definecolor{stringcolor}{rgb}{0.7,0.23,0.36}
\definecolor{backcolour}{rgb}{0.95,0.95,0.92}
\definecolor{keycolor}{rgb}{0.007,0.01,1.0}
\definecolor{itemcolor}{rgb}{0.01,0.0,0.49}

\hypersetup{
    colorlinks=true,
    linkcolor=blue,
    filecolor=blue,      
    urlcolor=blue,
}
 
\lstdefinestyle{mystyle}{
    %backgroundcolor=\color{backcolour},   
    commentstyle=\color{codegreen},
    keywordstyle=\color{keycolor},
    numberstyle=\tiny\color{codegray},
    stringstyle=\color{stringcolor},
    basicstyle=\footnotesize,
    breakatwhitespace=false,         
    breaklines=true,                 
    captionpos=b,                    
    keepspaces=true,                 
    numbers=left,                    
    numbersep=5pt,                  
    showspaces=false,                
    showstringspaces=false,
    showtabs=false,                  
    tabsize=2
}
 
\lstset{style=mystyle}

\lstdefinelanguage{Swift}{
  keywords={associatedtype, class, deinit, enum, extension, func, import, init, inout, internal, let, operator, private, protocol, public, static, struct, subscript, typealias, var, break, case, continue, default, defer, do, else, fallthrough, for, guard, if, in, repeat, return, switch, where, while, as, catch, dynamicType, false, is, nil, rethrows, super, self, Self, throw, throws, true, try, associativity, convenience, dynamic, didSet, final, get, infix, indirect, lazy, left, mutating, none, nonmutating, optional, override, postfix, precedence, prefix, Protocol, required, right, set, Type, unowned, weak, willSet},
  ndkeywords={class, export, boolean, throw, implements, import, this},
  sensitive=false,
  comment=[l]{//},
  morecomment=[s]{/*}{*/},
  morestring=[b]',
  morestring=[b]"
}

\lstset{emph={Int,count,abs,repeating,Array}, emphstyle=\color{itemcolor}}


\title{Week 05}

\date{\today}

\lstset{style=mystyle}

%%% BEGIN DOCUMENT
\begin{document}
\maketitle

\section{Preparation for Assignment}
If, and \textit{only if} you can truthfully assert the truthfulness of each statement below are you ready to start the exercises.
\subsection {Reading Comprehension Self-Check}
\begin{itemize}
\item  I know why it is \textbf{false} to say that \textit{divide-and-conquer} is a general algorithm design technique that starts solving a problem's instance by dividing it into several smaller instances, ideally of unequal size.
  \item  I know that the \href{run:../support_files/master_theorem.pdf}{Master Theorem} establishes the order of growth of the solutions to the general recurrence $T(n) = aT(n/b) + f(n)$ that the running time of many divide-and-conquer algorithms satisfies.
  \item  In addition, I know why it is \textbf{false} to say that the Master Theorem
    gives \textbf{explicit} solutions to this general recurrence.
  \item  I know that \textit{mergesort} and \textit{quicksort} are two divide-and-conquer sorting algorithms both of whose best-case time efficiency is \say{linear logarithmic}.
  \item  I know that \textit{decrease-and-conquer} might be considered a degenerate case of \textit{divide-and-conquer}, but it is better to consider them as two different design paradigms.
  \item  I know how to fill in the table below, that is, I can compare (by
    matching the correct phrase with an X in the correct table cell, only one X
    per each row and per each column) the orders of growth of two functions g(n) and f(n) when the ratio of g(n) to f(n), (i.e., g(n)/f(n), in the limit as n
    goes to infinity), approaches zero, or else some positive constant, or else
    infinity.
    
    $GR(a(n))$ is a function that returns the Growth Rate of a function. 
    
    \begin{tabular}{|c|c|c|c|}
    \hline
    $\lim_{x\to\infty} \frac{f(x)}{g(x)}$ & $GR(g(n))=GR(f(n))$&$GR(g(n)) >GR(f(n))$&$GR(g(n)) <GR(f(n))$\\
    \hline
    $0$&&&\\
    \hline
    $k\in\mathbb{N}, k\not=0$&&&\\
    \hline
    $\infty$&&&\\
    \hline
    \end{tabular}

\end{itemize}
\subsection{Memory Self-Check}

\subsubsection{Applying the Master Theorem}


By filling in the table (the first row is done for you), show that you know
  how to use the Master Theorem, page 197 and \href{run:../support_files/master_theorem.pdf}{Master Theorem}, to find the $\Theta$ order of growth for solutions of the following recurrence relations (where in every case, $T(1) = 1$):
  \\
\begin{tabular}{|c|l|c|c|c|c|c|}
\hline
    &$T(n) =$           & $a$ & $b$ & $d$ & $a$ <, =, or > $b^d$ & $\Theta$ \\
\hline
1. & $2T(n/2)+n-1$ & $2$ & $2$ & $1$ & $=$ & $n\log_2 n$ \\
\hline
2. & $2T(n/2)+2n+1$ &  &  & & &  \\
\hline
3. & $2T(n/2)+1$ &  &  & & &  \\
\hline
4. & $3T(n/3)+n^2+2n+1$ &  &  & & &  \\
\hline
5. & $4T(n/2)+n$ &  &  & & &  \\
\hline
6. & $4T(n/2)+n^2$ &  &  & & &  \\
\hline
7. & $4T(n/2)+n^3$ &  &  & & &  \\
\hline
\end{tabular}
 \section{Week 04 Exercises}
\subsection{ Exercise 1 on page 174} 
\subsection{Exercise 1 on page 181} 
\subsection{Exercise 5 on page 185} 
\subsection{Exercise 6 on page 186} 
\subsection{Exercise 2 on page 217}
\subsection{ Exercise 1 on page 191} 


\section{Week 04 Problems}
\subsection{Exercise 11 on page 175} 
\subsection{Exercise 9 on page 186} Make sure you write out the full mathematical proof.
\subsection{Exercise 11 on page 186}
\subsection{Exercise 12 on page 198}


\end{document}