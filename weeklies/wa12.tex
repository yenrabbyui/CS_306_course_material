

\documentclass[12pt]{amsart}
\usepackage{geometry} % see geometry.pdf on how to lay out the page. There's lots.
\geometry{a4paper} % or letter or a5paper or ... etc
\usepackage[T1]{fontenc}
\usepackage[latin9]{inputenc}
\usepackage{amsmath}
\usepackage{amsaddr}
\usepackage{dirtytalk}
\usepackage{float}
\usepackage{listings}
\usepackage{hyperref}
\usepackage{enumerate}
\usepackage{graphicx}

\usepackage{color}
 
\definecolor{codegreen}{rgb}{0,0.6,0}
\definecolor{codegray}{rgb}{0.5,0.5,0.5}
\definecolor{stringcolor}{rgb}{0.7,0.23,0.36}
\definecolor{backcolour}{rgb}{0.95,0.95,0.92}
\definecolor{keycolor}{rgb}{0.007,0.01,1.0}
\definecolor{itemcolor}{rgb}{0.01,0.0,0.49}

\hypersetup{
    colorlinks=true,
    linkcolor=blue,
    filecolor=blue,      
    urlcolor=blue,
}
 
\lstdefinestyle{mystyle}{
    %backgroundcolor=\color{backcolour},   
    commentstyle=\color{codegreen},
    keywordstyle=\color{keycolor},
    numberstyle=\tiny\color{codegray},
    stringstyle=\color{stringcolor},
    basicstyle=\footnotesize,
    breakatwhitespace=false,         
    breaklines=true,                 
    captionpos=b,                    
    keepspaces=true,                 
    numbers=left,                    
    numbersep=5pt,                  
    showspaces=false,                
    showstringspaces=false,
    showtabs=false,                  
    tabsize=2
}
 
\lstset{style=mystyle}

\lstdefinelanguage{Swift}{
  keywords={associatedtype, class, deinit, enum, extension, func, import, init, inout, internal, let, operator, private, protocol, public, static, struct, subscript, typealias, var, break, case, continue, default, defer, do, else, fallthrough, for, guard, if, in, repeat, return, switch, where, while, as, catch, dynamicType, false, is, nil, rethrows, super, self, Self, throw, throws, true, try, associativity, convenience, dynamic, didSet, final, get, infix, indirect, lazy, left, mutating, none, nonmutating, optional, override, postfix, precedence, prefix, Protocol, required, right, set, Type, unowned, weak, willSet},
  ndkeywords={class, export, boolean, throw, implements, import, this},
  sensitive=false,
  comment=[l]{//},
  morecomment=[s]{/*}{*/},
  morestring=[b]',
  morestring=[b]"
}

\lstset{emph={Int,count,abs,repeating,Array}, emphstyle=\color{itemcolor}}


\title{Week 12}

\date{\today}

\lstset{style=mystyle}

%%% BEGIN DOCUMENT
\begin{document}
\maketitle

\section{Preparation for Assignment}
If, and \textit{only if} you can truthfully assert the truthfulness of each statement below are you ready to start the exercises.
\subsection {Reading Comprehension Self-Check}
\begin{itemize}
\item  I know in what sense the power of algorithms is limited.
\item  I know that lower bounds $\Omega$ to many problems are known, i.e., no
    algorithm can undercut them.
\item  I can give at least two examples of problems with known lower bounds.
\item  I know that some problems cannot be fully solved.
\item  I know that there are problems for which algorithms are \textbf{not known} to
    exist.
\item  I know that there are problems for which algorithms are \textbf{known not} to
    exist.
\item  I know that many problems are considered intractable, which means
    infeasible to solve with current technology.
\item  I know that numerical algorithms face the limiting effects of
    truncation, roundoff, overflow, underflow and cancellation.

\end{itemize}
\subsection{Memory Self-Check}
I can, and have, explained to someone who is not a student in the Computer Science and Electrical Engineering, Computer Information Technology, or Mathematics departments what derivatives and integrals are and why they are important. 
 \section{Week 12 Exercises}
\subsection{ Exercise 5 on page 419}
\subsection{Exercise 10 on page 420} 
\subsection{Not in the Book}
Without doing an approximation, what is the derivative of 
$ y = 2x^3+1$?
\subsection{Not in the Book}
Without doing an approximation, what is the integral of  
$y = x^2+3x+2$?

\section{Week 12 Problems}
\subsection{Exercise 6 on page 420}
\subsection{Exercise 8 on page 420}



  


\end{document}
